\documentclass[12pt]{book}

\usepackage{amssymb}
\usepackage{ifthen}
\usepackage[table]{xcolor}
\usepackage{minitoc}
\usepackage{array}

\definecolor{yellow}{cmyk}{0,0,1,0}
\renewcommand{\arraystretch}{1.4}
\newcommand{\R}{\mathbb{R}}

\usepackage{colortbl}

% Page size
\setlength{\oddsidemargin}{-0.5in}
\setlength{\evensidemargin}{-0.5in}
\setlength{\textheight}{10.25in}
\setlength{\textwidth}{7.0in}
\setlength{\topmargin}{-1.35in}

\renewcommand{\arraycolsep}{1pt}


\input ../../../FLaTeX/color_flatex

\begin{document}
\pagestyle{empty}


\resetsteps % set all definitions

\begin{center}
	\FlaWorksheet
\end{center}


\resetsteps      % Reset all the commands to create a blank worksheet  


\resetsteps      % Reset all the commands to create a blank worksheet  

% Define the operation to be computed

\renewcommand{\operation}{ \left[ y \right] := \mbox{\sc sym\_axpy\_unb\_var1}( A, x, y ) }

\renewcommand{\routinename}{\operation}

% Step 1a: Precondition 

\renewcommand{\precondition}{
	y = \widehat{y}
}

% Step 1b: Postcondition 

\renewcommand{\postcondition}{ 
	\left[y \right]
	=
	\mbox{sym\_axpy}( A, x, \widehat{y} )
}

% Step 2: Invariant 
% Note: Right-hand side of equalities must be updated appropriately

\renewcommand{\invariant}{
	\FlaTwoByOne{y_T}
	{y_B} = 
	\FlaTwoByOne{\widehat{y}_T}
	{\widehat{y}_B}
}

% Step 3: Loop-guard 

\renewcommand{\guard}{
	m( A_{TL} ) < m( A )
}

% Step 4: Initialize 

\renewcommand{\partitionings}{
	$
	A \rightarrow
	\FlaTwoByTwo{A_{TL}}{A_{TR}}
	{A_{BL}}{A_{BR}}
	$
	,
	$
	x \rightarrow
	\FlaTwoByOne{x_{T}}
	{x_{B}}
	$
	,
	$
	y \rightarrow
	\FlaTwoByOne{y_{T}}
	{y_{B}}
	$
}

\renewcommand{\partitionsizes}{
	$ A_{TL} $ is $ 0 \times 0 $,
	$ x_T $ has $ 0 $ rows,
	$ y_T $ has $ 0 $ rows
}

% Step 5a: Repartition the operands 

\renewcommand{\repartitionings}{
	$  \FlaTwoByTwo{A_{TL}}{A_{TR}}
	{A_{BL}}{A_{BR}}
	\rightarrow
	\FlaThreeByThreeBR{A_{00}}{a_{01}}{A_{02}}
	{a_{10}^T}{\alpha_{11}}{a_{12}^T}
	{A_{20}}{a_{21}}{A_{22}}
	$,
	$  \FlaTwoByOne{ x_T }
	{ x_B }
	\rightarrow
	\FlaThreeByOneB{x_0}
	{\chi_1}
	{x_2}
	$
	,
	$  \FlaTwoByOne{ y_T }
	{ y_B }
	\rightarrow
	\FlaThreeByOneB{y_0}
	{\psi_1}
	{y_2}
	$
}

\renewcommand{\repartitionsizes}{
	$ \alpha_{11} $ is $ 1 \times 1 $,
	$ \chi_1 $ has $ 1 $ row,
	$ \psi_1 $ has $ 1 $ row}

% Step 5b: Move the double lines 

\renewcommand{\moveboundaries}{
	$  \FlaTwoByTwo{A_{TL}}{A_{TR}}
	{A_{BL}}{A_{BR}}
	\leftarrow
	\FlaThreeByThreeTL{A_{00}}{a_{01}}{A_{02}}
	{a_{10}^T}{\alpha_{11}}{a_{12}^T}
	{A_{20}}{a_{21}}{A_{22}}
	$,
	$  \FlaTwoByOne{ x_T }
	{ x_B }
	\leftarrow
	\FlaThreeByOneT{x_0}
	{\chi_1}
	{x_2}
	$
	,
	$  \FlaTwoByOne{ y_T }
	{ y_B }
	\leftarrow
	\FlaThreeByOneT{y_0}
	{\psi_1}
	{y_2}
	$
}

% Step 6: State after repartitioning
% Note: The below needs editing!!!

\renewcommand{\beforeupdate}{
	\FlaThreeByOneB{y_0}
	{\psi_1}
	{y_2}
}

% Step 7: State after moving of double lines
% Note: The below needs editing!!!

\renewcommand{\afterupdate}{
	\FlaThreeByOneT{y_0}
	{\psi_1}
	{y_2}
}

% Step 8: Insert the updates required to change the 
%         state from that given in Step 6 to that given in Step 7
% Note: The below needs editing!!!

\renewcommand{\update}{
	$
	\begin{array}{l}
	\mbox{update line 1} \\ 
	\mbox{\ \ \ \ :} \\ 
	\mbox{update line n} \\ 
	\end{array}
	$
}

\begin{figure}[p]
\begin{center}
\FlaWorksheetOne
\end{center}
\end{figure}

\newpage 

\begin{figure}[p]
\begin{center}
\FlaWorksheetTwo
\end{center}
\end{figure}


\newpage 

\begin{figure}[p]
\begin{center}
\FlaWorksheetThree
\end{center}
\end{figure}

\newpage 

\begin{figure}[p]
\begin{center}
\FlaWorksheetFour    
\end{center}
\end{figure}

\newpage 

\begin{figure}[p]
\begin{center}
\FlaWorksheetFive
\end{center}
\end{figure}

\newpage 

\begin{figure}[p]
\begin{center}
\FlaWorksheetSix
\end{center}
\end{figure}

\newpage 

\begin{figure}[p]
\begin{center}
\FlaWorksheetSeven
\end{center}
\end{figure}

\newpage 

\begin{figure}[p]
\begin{center}
\FlaWorksheetEight
\end{center}
\end{figure}

\newpage 

\begin{figure}[p]
\begin{center}
\FlaWorksheetNine
\end{center}
\end{figure}

\newpage 

\begin{figure}[p]
	\begin{center} 
		\FlaWorksheet
	\end{center}
\end{figure}

\newpage 

\begin{figure}[p]
	\begin{center}
		\FlaAlgorithm 
	\end{center}
\end{figure}

\end{document}