\documentclass[12pt]{book}

\usepackage{amssymb}
\usepackage{ifthen}
\usepackage[table]{xcolor}
\usepackage{minitoc}
\usepackage{array}

\definecolor{yellow}{cmyk}{0,0,1,0}
\renewcommand{\arraystretch}{1.4}
\newcommand{\R}{\mathbb{R}}

\usepackage{colortbl}

% Page size
\setlength{\oddsidemargin}{-0.5in}
\setlength{\evensidemargin}{-0.5in}
\setlength{\textheight}{10.25in}
\setlength{\textwidth}{7.0in}
\setlength{\topmargin}{-1.35in}

\renewcommand{\arraycolsep}{1pt}


\input color_flatex

\begin{document}
\pagestyle{empty}


\resetsteps % set all definitions

\begin{center}
	\FlaWorksheet
\end{center}


\resetsteps      % Reset all the commands to create a blank worksheet  

% Define the operation to be computed

\renewcommand{\operation}{ \left[ C \right] := \mbox{\sc syrk2\_unb\_var1}( A, B, C ) }

\renewcommand{\routinename}{\operation}

% Step 1a: Precondition 

\renewcommand{\precondition}{
	C = \widehat{C}
}

% Step 1b: Postcondition 

\renewcommand{\postcondition}{ 
	\left[C \right]
	=
	\mbox{syrk2}( A, B, \widehat{C} )
}

% Step 2: Invariant 
% Note: Right-hand side of equalities must be updated appropriately

\renewcommand{\invariant}{
	\FlaTwoByTwo{C_{TL}}{C_{TR}}
	{C_{BL}}{C_{BR}} =
	\FlaTwoByTwo{\widehat{C}_{TL} + A_TB_T^T + B_TA_T^T}{\widehat{C}_{TR}}
	{\widehat{C}_{BL}}{\widehat{C}_{BR}}
}

% Step 3: Loop-guard 

\renewcommand{\guard}{
	m( A_T ) < m( A )
}

% Step 4: Initialize 

\renewcommand{\partitionings}{
	$
	A \rightarrow
	\FlaTwoByOne{A_{T}}
	{A_{B}}
	$
	,
	$
	B \rightarrow
	\FlaTwoByOne{B_{T}}
	{B_{B}}
	$
	,
	$
	C \rightarrow
	\FlaTwoByTwo{C_{TL}}{C_{TR}}
	{C_{BL}}{C_{BR}}
	$
}

\renewcommand{\partitionsizes}{
	$ A_T $ has $ 0 $ rows,
	$ B_T $ has $ 0 $ rows,
	$ C_{TL} $ is $ 0 \times 0 $
}

% Step 5a: Repartition the operands 

\renewcommand{\repartitionings}{
	$  \FlaTwoByOne{ A_T }
	{ A_B }
	\rightarrow
	\FlaThreeByOneB{A_0}
	{a_1^T}
	{A_2}
	$
	,
	$  \FlaTwoByOne{ B_T }
	{ B_B }
	\rightarrow
	\FlaThreeByOneB{B_0}
	{b_1^T}
	{B_2}
	$
	,
	$  \FlaTwoByTwo{C_{TL}}{C_{TR}}
	{C_{BL}}{C_{BR}}
	\rightarrow
	\FlaThreeByThreeBR{C_{00}}{c_{01}}{C_{02}}
	{c_{10}^T}{\gamma_{11}}{c_{12}^T}
	{C_{20}}{c_{21}}{C_{22}}
	$}

\renewcommand{\repartitionsizes}{
	$ a_1 $ has $ 1 $ row,
	$ b_1 $ has $ 1 $ row,
	$ \gamma_{11} $ is $ 1 \times 1 $}

% Step 5b: Move the double lines 

\renewcommand{\moveboundaries}{
	$  \FlaTwoByOne{ A_T }
	{ A_B }
	\leftarrow
	\FlaThreeByOneT{A_0}
	{a_1^T}
	{A_2}
	$
	,
	$  \FlaTwoByOne{ B_T }
	{ B_B }
	\leftarrow
	\FlaThreeByOneT{B_0}
	{b_1^T}
	{B_2}
	$
	,
	$  \FlaTwoByTwo{C_{TL}}{C_{TR}}
	{C_{BL}}{C_{BR}}
	\leftarrow
	\FlaThreeByThreeTL{C_{00}}{c_{01}}{C_{02}}
	{c_{10}^T}{\gamma_{11}}{c_{12}^T}
	{C_{20}}{c_{21}}{C_{22}}
	$}

% Step 6: State after repartitioning
% Note: The below needs editing!!!

\renewcommand{\beforeupdate}{
	\FlaThreeByThreeBR{C_{00}}{c_{01}}{C_{02}}
	{c_{10}^T}{\gamma_{11}}{c_{12}^T}
	{C_{20}}{c_{21}}{C_{22}}
	=
	\FlaThreeByThreeBR
	{A_0B_0^T + B_0A_0^T + \widehat{C_{00}}} {\widehat{c_{01}}} {\widehat{C_{02}}}
	{\widehat{c_{10}^T}} {\widehat{\gamma_{11}}} {\widehat{c_{12}^T}}
	{\widehat{C_{20}}} {\widehat{c_{21}}} {\widehat{C_{22}}}
}

% Step 7: State after moving of double lines
% Note: The below needs editing!!!

\renewcommand{\afterupdate}{
	\FlaThreeByThreeTL
	{C_{00}}{c_{01}}{C_{02}}
	{c_{10}^T}{\gamma_{11}}{c_{12}^T}
	{C_{20}}{c_{21}}{C_{22}}
	=
	\FlaThreeByThreeTL
	{A_0B_0^T + B_0A_0^T + \widehat{C_{00}}} {A_0\beta_1 + B_0\alpha_1 + \widehat{c_{01}}} {\widehat{C_{02}}}
	{\alpha_1^TB_0^T + \beta_1^TA_0^T + \widehat{c_{10}^T}} {\alpha_1^T\beta_1 + \beta_1^T\alpha_1 + \widehat{\gamma_{11}}} {\widehat{c_{12}^T}}
	{\widehat{C_{20}}} {\widehat{c_{21}}} {\widehat{C_{22}}}
}

% Step 8: Insert the updates required to change the 
%         state from that given in Step 6 to that given in Step 7
% Note: The below needs editing!!!

\renewcommand{\update}{
	$
	c_{01} = A_0\beta_1 + B_0\alpha_1 + c_{01}	
	c_{10}^T = c_{01}^T
	\gamma_{11} = \alpha_1^T\beta_1 + \beta_1^T\alpha_1 + \gamma_{11}
	$
}


\begin{figure}[p]
\begin{center} 
\FlaWorksheet
\end{center}
\end{figure}

\newpage 

\begin{figure}[p]
\begin{center}
\FlaAlgorithm 
\end{center}
\end{figure}

\newpage 

\begin{figure}[p]
\begin{center}
\FlaWorksheetOne
\end{center}
\end{figure}

\newpage 

\begin{figure}[p]
\begin{center}
\FlaWorksheetTwo
\end{center}
\end{figure}



\newpage 

\begin{figure}[p]
\begin{center}
\FlaWorksheetThree
\end{center}
\end{figure}

\newpage 

\begin{figure}[p]
\begin{center}
\FlaWorksheetFour    
\end{center}
\end{figure}

\newpage 

\begin{figure}[p]
\begin{center}
\FlaWorksheetFive
\end{center}
\end{figure}

\newpage 

\begin{figure}[p]
\begin{center}
\FlaWorksheetSix
\end{center}
\end{figure}

\newpage 

\begin{figure}[p]
\begin{center}
\FlaWorksheetSeven
\end{center}
\end{figure}

\newpage 

\begin{figure}[p]
\begin{center}
\FlaWorksheetEight
\end{center}
\end{figure}

\newpage 

\begin{figure}[p]
\begin{center}
\FlaWorksheetNine
\end{center}
\end{figure}

\newpage 

\begin{figure}[p]
\begin{center}
\FlaAlgorithm 
\end{center}
\end{figure}

\end{document}