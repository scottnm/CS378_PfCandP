\documentclass[12pt]{book}

\usepackage{amssymb}
\usepackage{ifthen}
\usepackage[table]{xcolor}
\usepackage{minitoc}
\usepackage{array}

\definecolor{yellow}{cmyk}{0,0,1,0}
\renewcommand{\arraystretch}{1.4}
\newcommand{\R}{\mathbb{R}}

\usepackage{colortbl}

% Page size
\setlength{\oddsidemargin}{-0.5in}
\setlength{\evensidemargin}{-0.5in}
\setlength{\textheight}{10.25in}
\setlength{\textwidth}{7.0in}
\setlength{\topmargin}{-1.35in}

\renewcommand{\arraycolsep}{1pt}


\input color_flatex

\begin{document}
\pagestyle{empty}


\resetsteps % set all definitions

\begin{center}
	\FlaWorksheet
\end{center}

\resetsteps      % Reset all the commands to create a blank worksheet  

% Define the operation to be computed

\renewcommand{\operation}{ \left[ C \right] := \mbox{\sc syr2k\_unb\_var6}( A, B, C ) }

\renewcommand{\routinename}{\operation}

% Step 1a: Precondition 

\renewcommand{\precondition}{
  C = \widehat{C}
}

% Step 1b: Postcondition 

\renewcommand{\postcondition}{ 
  \left[C \right]
  =
  \mbox{syr2k}( A, B, \widehat{C} )
}

% Step 2: Invariant 
% Note: Right-hand side of equalities must be updated appropriately

\renewcommand{\invariant}{
	C = A_RB_R^T + B_RA_R^T + \widehat{C}
}

% Step 3: Loop-guard 

\renewcommand{\guard}{
  n( A_R ) < n( A )
}

% Step 4: Initialize 

\renewcommand{\partitionings}{
  $
  A \rightarrow
  \FlaOneByTwo{A_L}{A_R}
  $
,
  $
  B \rightarrow
  \FlaOneByTwo{B_L}{B_R}
  $
}

\renewcommand{\partitionsizes}{
$ A_R $ has $ 0 $ columns,
$ B_R $ has $ 0 $ columns
}

% Step 5a: Repartition the operands 

\renewcommand{\repartitionings}{
$  \FlaOneByTwo{A_L}{A_R}
\rightarrow  \FlaOneByThreeL{A_0}{a_1}{A_2}
$
,
$  \FlaOneByTwo{B_L}{B_R}
\rightarrow  \FlaOneByThreeL{B_0}{b_1}{B_2}
$
}

\renewcommand{\repartitionsizes}{
$ a_1 $ has $ 1 $ column,
$ b_1 $ has $ 1 $ column}

% Step 5b: Move the double lines 

\renewcommand{\moveboundaries}{
$  \FlaOneByTwo{A_L}{A_R}
\leftarrow  \FlaOneByThreeR{A_0}{a_1}{A_2}
$
,
$  \FlaOneByTwo{B_L}{B_R}
\leftarrow  \FlaOneByThreeR{B_0}{b_1}{B_2}
$
}

% Step 6: State after repartitioning
% Note: The below needs editing!!!

\renewcommand{\beforeupdate}{
	C = A_2B_2^T + B_2A_2^T + \widehat{C}
}

% Step 7: State after moving of double lines
% Note: The below needs editing!!!

\renewcommand{\afterupdate}{
	C = A_2B_2^T + B_2A_2^T + a_1b_1^T + b_1a_1^T + \widehat{C}
}

% Step 8: Insert the updates required to change the 
%         state from that given in Step 6 to that given in Step 7
% Note: The below needs editing!!!

\renewcommand{\update}{
$
  C := a_1b_1^T + b_1a_1^T + C
$
}

\begin{figure}[p]
\begin{center} 
\FlaWorksheet
\end{center}
\end{figure}

\newpage 

\begin{figure}[p]
\begin{center}
\FlaAlgorithm 
\end{center}
\end{figure}

\newpage 

\begin{figure}[p]
\begin{center}
\FlaWorksheetOne
\end{center}
\end{figure}

\newpage 

\begin{figure}[p]
\begin{center}
\FlaWorksheetTwo
\end{center}
\end{figure}



\newpage 

\begin{figure}[p]
\begin{center}
\FlaWorksheetThree
\end{center}
\end{figure}

\newpage 

\begin{figure}[p]
\begin{center}
\FlaWorksheetFour    
\end{center}
\end{figure}

\newpage 

\begin{figure}[p]
\begin{center}
\FlaWorksheetFive
\end{center}
\end{figure}

\newpage 

\begin{figure}[p]
\begin{center}
\FlaWorksheetSix
\end{center}
\end{figure}

\newpage 

\begin{figure}[p]
\begin{center}
\FlaWorksheetSeven
\end{center}
\end{figure}

\newpage 

\begin{figure}[p]
\begin{center}
\FlaWorksheetEight
\end{center}
\end{figure}

\newpage 

\begin{figure}[p]
\begin{center}
\FlaWorksheetNine
\end{center}
\end{figure}

\newpage 

\begin{figure}[p]
\begin{center}
\FlaAlgorithm 
\end{center}
\end{figure}

\end{document}