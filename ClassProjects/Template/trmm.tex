\documentclass[12pt]{book}

\usepackage{amssymb}
\usepackage{ifthen}
\usepackage[table]{xcolor}
\usepackage{minitoc}
\usepackage{array}

\definecolor{yellow}{cmyk}{0,0,1,0}
\renewcommand{\arraystretch}{1.4}
\newcommand{\R}{\mathbb{R}}

\usepackage{colortbl}

% Page size
\setlength{\oddsidemargin}{-0.5in}
\setlength{\evensidemargin}{-0.5in}
\setlength{\textheight}{10.25in}
\setlength{\textwidth}{7.0in}
\setlength{\topmargin}{-1.35in}

\renewcommand{\arraycolsep}{1pt}


\input color_flatex

\begin{document}
	\pagestyle{empty}
	
	
	\resetsteps      % Reset all the commands to create a blank worksheet  
	
	% Define the operation to be computed
	
	\renewcommand{\operation}{ \left[ B \right] := \mbox{\sc trmm\_blk\_var1}( L, B ) }
	
	\renewcommand{\routinename}{\operation}
	
	% Step 1a: Precondition 
	
	\renewcommand{\precondition}{
		B = \widehat{B}
	}
	
	% Step 1b: Postcondition 
	
	\renewcommand{\postcondition}{ 
		\left[B \right]
		=
		\mbox{trmm}( L, \widehat{B} )
	}
	
	% Step 2: Invariant 
	% Note: Right-hand side of equalities must be updated appropriately
	
	\renewcommand{\invariant}{
		\FlaTwoByOne{B_T}
		{B_B} = 
		\FlaTwoByOne{\widehat{B}_T}
		{\widehat{B}_B}
	}
	
	% Step 3: Loop-guard 
	
	\renewcommand{\guard}{
		m( L_{BR} ) < m( L )
	}
	
	% Step 4: Initialize 
	
	\renewcommand{\partitionings}{
		$
		L \rightarrow
		\FlaTwoByTwo{L_{TL}}{L_{TR}}
		{L_{BL}}{L_{BR}}
		$
		,
		$
		B \rightarrow
		\FlaTwoByOne{B_{T}}
		{B_{B}}
		$
	}
	
	\renewcommand{\partitionsizes}{
		$ L_{BR} $ is $ 0 \times 0 $,
		$ B_B $ has $ 0 $ rows
	}
	
	% Step 5a: Repartition the operands 
	
	\renewcommand{\blocksize}{b}
	
	\renewcommand{\repartitionings}{
		$  \FlaTwoByTwo{L_{TL}}{L_{TR}}
		{L_{BL}}{L_{BR}}
		\rightarrow
		\FlaThreeByThreeTL{L_{00}}{L_{01}}{L_{02}}
		{L_{10}}{L_{11}}{L_{12}}
		{L_{20}}{L_{21}}{L_{22}}
		$,
		$  \FlaTwoByOne{ B_T }
		{ B_B }
		\rightarrow
		\FlaThreeByOneT{B_0}
		{B_1}
		{B_2}
		$
	}
	
	\renewcommand{\repartitionsizes}{
		$ L_{11} $ is $ b \times b $,
		$ B_1 $ has $ b $ rows}
	
	% Step 5b: Move the double lines 
	
	\renewcommand{\moveboundaries}{
		$  \FlaTwoByTwo{L_{TL}}{L_{TR}}
		{L_{BL}}{L_{BR}}
		\leftarrow
		\FlaThreeByThreeBR{L_{00}}{L_{01}}{L_{02}}
		{L_{10}}{L_{11}}{L_{12}}
		{L_{20}}{L_{21}}{L_{22}}
		$,
		$  \FlaTwoByOne{ B_T }
		{ B_B }
		\leftarrow
		\FlaThreeByOneB{B_0}
		{B_1}
		{B_2}
		$
	}
	
	% Step 6: State after repartitioning
	% Note: The below needs editing!!!
	
	\renewcommand{\beforeupdate}{
		\FlaThreeByOneT{B_0}
		{B_1}
		{B_2}
		=
		\FlaThreeByOneT
		{\widehat{B_0}}
		{B_1}
		{L_{22}\widehat{B_2}}
	}
	
	% Step 7: State after moving of double lines
	% Note: The below needs editing!!!
	
	\renewcommand{\afterupdate}{
		\FlaThreeByOneB{B_0}
		{B_1}
		{B_2}
		=
		\FlaThreeByOneB
		{\widehat{B_0}}
		{L_{11}\widehat{B_1}}
		{L_{21}\widehat{B_1} + L_{22}\widehat{B_2}}
	}
	
	% Step 8: Insert the updates required to change the 
	%         state from that given in Step 6 to that given in Step 7
	% Note: The below needs editing!!!
	
	\renewcommand{\update}{
		$
		
		B_2 := L_{21}B_1 + B_2
		
		B_1 := L_{11}B_1
		
		$
		
	}
	
	
	
	\begin{figure}[p]
		\begin{center} 
			\FlaWorksheet
		\end{center}
	\end{figure}
	
	\newpage 
	
	\begin{figure}[p]
		\begin{center}
			\FlaAlgorithm 
		\end{center}
	\end{figure}
	
	\newpage 
	
	\begin{figure}[p]
		\begin{center}
			\FlaWorksheetOne
		\end{center}
	\end{figure}
	
	\newpage 
	
	\begin{figure}[p]
		\begin{center}
			\FlaWorksheetTwo
		\end{center}
	\end{figure}
	
	
	
	\newpage 
	
	\begin{figure}[p]
		\begin{center}
			\FlaWorksheetThree
		\end{center}
	\end{figure}
	
	\newpage 
	
	\begin{figure}[p]
		\begin{center}
			\FlaWorksheetFour    
		\end{center}
	\end{figure}
	
	\newpage 
	
	\begin{figure}[p]
		\begin{center}
			\FlaWorksheetFive
		\end{center}
	\end{figure}
	
	\newpage 
	
	\begin{figure}[p]
		\begin{center}
			\FlaWorksheetSix
		\end{center}
	\end{figure}
	
	\newpage 
	
	\begin{figure}[p]
		\begin{center}
			\FlaWorksheetSeven
		\end{center}
	\end{figure}
	
	\newpage 
	
	\begin{figure}[p]
		\begin{center}
			\FlaWorksheetEight
		\end{center}
	\end{figure}
	
	\newpage 
	
	\begin{figure}[p]
		\begin{center}
			\FlaWorksheetNine
		\end{center}
	\end{figure}
	
	\newpage 
	
	\begin{figure}[p]
		\begin{center}
			\FlaAlgorithm 
		\end{center}
	\end{figure}
	
\end{document}